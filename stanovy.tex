\documentclass[a4paper]{article}

\usepackage[english,czech]{babel}
\usepackage[T1]{fontenc}
\usepackage[utf8]{inputenc}
\usepackage{cmap}
\usepackage{lmodern}
\usepackage[final]{pdfpages}
\usepackage{idxlayout}

\ifpdf
    \usepackage[protrusion,expansion,step=1]{microtype}
\else
    \usepackage[protrusion]{microtype}
\fi

\usepackage{makeidx}
\usepackage{units}

\makeindex

\usepackage[
bookmarks=true,
colorlinks=true,
breaklinks=true,
urlcolor=black,
citecolor=black,
linkcolor=black,
unicode=true,
]
{hyperref}

\usepackage[square, numbers]{natbib}
\usepackage{url}


\RequirePackage[left=1in,right=1in,top=1in,bottom=1in]{geometry}
\usepackage{enumerate}


\newcommand{\halfspace}{\hskip 0.4ex}
\newcommand{\mlname}{Mladí lidovci, z.\halfspace{}s.}


\usepackage[explicit]{titlesec}
\titleformat{\section}
  {\normalfont\Large\filcenter}{}{0em}{\shortstack{Čl. \arabic{section} \\ \vspace*{0.2em} \\ \halfspace\hspace*{ 0.3em}#1}}
\titleformat{name=\section,numberless}
  {\normalfont\Large\filcenter}{}{0em}{\MakeUppercase{#1}}

\renewcommand*{\thesection}{\Roman{section}}



\title{\normalfont\Huge{Stanovy \mlname}}
\date{}
\author{}

\begin{document}

\maketitle

\section{Základní ustanovení}
    \begin{enumerate}
    \item \mlname{} jsou spolkem řádně zapsaným ve spolkovém
        rejstříku v~souladu se zákonem č.~304/2013~Sb.

    \item Úplný název spolku je \uv{\mlname} (dále také
        spolek). V~případech, kdy je možné tak učinit, spolek může
        vystupovat pod názvem \uv{Mladí lidovci} nebo zkratkou \uv{ML}.
        Anglický překlad názvu spolku je \uv{Youth of People's Party},
        zkratkou \uv{YPP}, německý překlad je \uv{Junge Volksparteiler},
        zkratkou \uv{JV}.

    \item Existence \mlname{} je založena na přijetí odpovědnosti za
        konstruktivní utváření budoucnosti Evropy. Činí tak s~vědomím
        kulturního a sociálního bohatství vytvořeného během staletí trvajících
        křesťanských dějin Evropy, které nikde jinde ani později již nenašlo
        obdoby. Proto se svojí prací zasazuje o~uchování společného duchovního
        dědictví a křesťanské zásady si bere za základ své činnosti. \mlname{}
        úzce spolupracující s~partnerskými občanskými spolky v~České
        republice i zahraničními organizacemi, nabízí tak solidní podporu
        nejen křesťanům, ale všem mladým lidem, zejména politicky angažovaným,
        kteří se chtějí aktivně zasazovat o~lepší budoucnost České republiky
        a Evropy.

    \item Adresa sídla spolku: Palác Charitas, Karlovo náměstí 317/5,
        128~01~Praha~2.
    \end{enumerate}



\section{Poslání, cíle a činnost}
    \begin{enumerate}
    \item Posláním spolku je vytváření prostředí stimulujícího osobnostní,
        profesní, sociální, politický a duchovní růst mladého člověka.

    \item Zvláštní zřetel je kladen na růst společensko-politický.

    \item Cíle spolku, orientované na mladé lidi, studenty a absolventy
        vysokých škol, jsou zejména následující:
        \begin{enumerate}
        \item podporovat vzdělání v~otázkách křesťansko-konzervativní
            politiky, s~důrazem ukotvení hodnot založených na evropské
            křesťanské morálce,

        \item naplňovat společenské potřeby cílové skupiny,

        \item podporovat vzájemnou úctu, ohleduplnost a nezištnou službu
            v~rámci aktivního působení ve veřejném životě,

        \item iniciovat rozvoj schopností a dovedností umožňujících hlubší
            pochopení politické morální odpovědnosti,

        \item podporovat zvýšení informovanosti o~politickém a kulturním
            dění ve světě, obzvláště v~České republice a v~Evropské unii,

        \item pomáhat rozvíjet schopnost zaujímat a vyjadřovat stanoviska
            ke společenským otázkám.
        \end{enumerate}

    \item Činnost spolku spočívá zejména v~následujícím:
        \begin{enumerate}
        \item vytváření společenství mladých lidí se zájmem o~politické dění,

        \item organizování a zprostředkovávání vzdělávacích akcí, přednášek,
            seminářů, stáží a jiných aktivit potřebných k formaci osobnosti,

        \item pořádání kulturních a společenských akcí.
        \end{enumerate}
    \end{enumerate}



\section{Členství}
    \begin{enumerate}
    \item Členem spolku může být:
        \begin{enumerate}
             \item osoba starší 15 let a mladší 35 let věku a
             \item osoba, která není členem žádné politické strany nebo její mládežnické organizace, s výjimkou
                \begin{enumerate}
                     \item politické strany KDU-ČSL a její mládežnické organizace,
                     \item zahraniční politické strany založené na křesťansko-demokratických
                        principech a její mládežnické organizace.
                \end{enumerate}
        \end{enumerate}

    \item K~přijetí za člena spolku dochází na základě podání přihlášky a
        jejím schválením předsednictvem spolku. Přihláška se podává
        prostřednictvím krajského výboru, který ji předá předsednictvu
        spolku spolu se svým stanoviskem.

    \item O~zařazení člena do právě jedné krajské organizace rozhoduje
        předsednictvo spolku. Člen může být členem nejvýše v~jedné okresní a v~jedné místní
        organizaci.

    \item Pokud je přihláška o~členství předsednictvem spolku zamítnuta,
        musí být uchazeč písemně informován nejpozději do 30 pracovních dnů.
        Uchazeč o~členství, jehož přihláška byla zamítnuta, může podat
        přihlášku znovu nejdříve po uplynutí 6 měsíců ode dne zamítnutí
        přihlášky.

    \item Členství ve spolku zaniká:
        \begin{enumerate}
        \item úmrtím,
        \item dovršením věku 35 let,
        \item písemným prohlášením o vystoupení doručeným předsednictvu
            spolku,
        \item porušením podmínky členství dle ustanovení Čl.~3.~odst.~1.,
        \item nezaplacením členského příspěvku,
        \item vyloučením.
        \end{enumerate}


    \item O~vyloučení člena rozhoduje předsednictvo spolku. Vyloučit člena
        spolku lze pouze pro:
        \begin{enumerate}
        \item hrubé porušení stanov,
        \item dlouhodobé zanedbávání členských povinností,
        \item hrubé a zaviněné porušení povinností při hospodaření s~majetkem
            spolku,
        \item spáchání úmyslného trestného činu, za který byl odsouzen,
        \item jednání, které poškozuje spolek.
        \end{enumerate}

    \item Proti rozhodnutí o~vyloučení se může člen spolku odvolat ke
        kontrolní komisi do 15 dnů ode dne doručení písemného vyhotovení
        rozhodnutí. V~případě podání odvolání vyloučeným členem se účinnost
        rozhodnutí o~vyloučení pozastavuje až do rozhodnutí kontrolní komise.

    \item Člen spolku má právo:
        \begin{enumerate}
        \item účastnit se jednání schůze místní, okresní a krajské organizace
        a členských schůzí, jichž je členem,

        \item volit a být volen do orgánů spolku,

        \item být informován o přijatých usneseních orgánů spolku,

        \item předkládat všem orgánům spolku náměty, stížnosti a připomínky
        a být do devadesáti dnů vyrozuměn o řešení,

        \item svobodně vyjadřovat a obhajovat své názory a předávat je
        k rozhodnutí voleným orgánům spolku,

        \item v případě sporu se odvolávat ke kontrolní komisi,

        \item účastnit se zasedání orgánů spolku, jestliže se rozhoduje
        o činnosti nebo jednání daného člena spolku.
        \end{enumerate}

    \item Člen spolku je povinen:
        \begin{enumerate}
        \item prosazovat cíle spolku,

        \item řídit se stanovami spolku,

        \item plnit usnesení a rozhodnutí orgánů spolku,

        \item plnit převzaté úkoly a nést za ně odpovědnost,

        \item platit stanovené členské příspěvky,

        \item neprodleně oznamovat předsedovi krajského výboru
        změny kontaktních údajů, které jsou předmětem členské evidence.
        \end{enumerate}

    \item Členům, kterým zaniklo členství podle Čl.~3.~odst.~5.~písm.~b), může být uděleno celostátním
        výborem na návrh předsednictva čestné členství. Na čestné členství se neváží žádná práva a
        povinnosti, neboť se nejedná o členství ve smyslu Čl.~3.

    \item Členům a bývalým členům, kteří v minulosti vykonávali funkci
        předsedy spolku, může být celostátním sjezdem na návrh celostátního výboru uděleno čestné
        předsednictví spolku. Na čestné předsednictví se neváží žádná práva a povinnosti, neboť se
        nejedná o předsednictví ve smyslu Čl.~4.~odst.~5.

    \end{enumerate}



\section{Organizační struktura a orgány spolku}
    \begin{enumerate}
    \item Organizační struktura spolku se skládá
        \begin{enumerate}
        \item z~celostátní organizace (tj. všech členů spolku),
        \item krajských organizací,
        \item okresních organizací,
        \item místních organizací.
        \end{enumerate}

    \item Orgány:
        \begin{enumerate}
        \item Orgány celostátní organizace jsou celostátní sjezd,
            celostátní výbor,
            předsednictvo spolku a kontrolní komise.
        \item Orgány krajské organizace jsou krajská konference a krajský
            výbor.
        \item Orgány okresní organizace jsou okresní konference a okresní výbor.
        \item Orgány místní organizace jsou výroční členská schůze a
            předsednictvo místní organizace.
        \item Orgány spolku jsou usnášeníschopné za přítomnosti nadpolovičního počtu
            členů, není-li stanoveno jinak.
        \item Funkční období volených orgánů spolku je jednoleté.
        \end{enumerate}

    \item Celostátní sjezd:
        \begin{enumerate}
        \item Celostátní sjezd je nejvyšší orgán spolku a schází se nejméně
            jednou za rok.

        \item Celostátní sjezd svolává předsednictvo spolku, nebo
            \nicefrac{1}{3} členů spolku svým podpisem, a to nejméně 20 dnů
            před termínem konání.

        \item Delegáty celostátního sjezdu s~hlasovacím právem jsou členové
            předsednictva spolku, členové celostátního výboru spolku,
            předsedové krajských výborů a zástupci
            krajských organizací zvolení na krajské konferenci. Klíč pro počet
            delegátů z~jednotlivých krajů určuje předsednictvo spolku
            v~usnesení o~svolání celostátního sjezdu tak, aby každá krajská
            organizace měla na celostátním sjezdu alespoň jednoho voleného
            delegáta.

        \item Výlučnou pravomocí celostátního sjezdu je volba a odvolání
            předsednictva a celostátního výboru spolku, změna stanov a volba
            a odvolání členů kontrolní komise. Volba do těchto orgánů
            spolku je tajná.

        \item Celostátní sjezd rozhoduje o~výši členských příspěvků,
            nesvěří-li tuto pravomoc celostátnímu výboru.

        \item Celostátní sjezd může rozhodnout o~zrušení spolku nebo jeho
            sloučení s~jiným spolkem, a to \nicefrac{2}{3} většinou hlasů
            z~počtu delegátů.
        \end{enumerate}

    \item Celostátní výbor:
        \begin{enumerate}
        \item Celostátní výbor je nejvyšším orgánem spolku mezi 
            celostátními sjezdy.

        \item Schází se nejméně čtyřikrát za rok.

        \item Členy celostátního výboru jsou členové předsednictva
            spolku a členové zvolení celostátním sjezdem.
            Celostátní sjezd určí počet volených členů a způsob volby
            tak, aby byl z~každé krajské organizace zvolen alespoň
            jeden člen.

        \item Podmínkou usnášeníschopnosti celostátního výboru
            je nadpoloviční většina hlasů přítomných členů.

        \item Celostátní výbor je svolán předsedou spolku
            nebo na žádost \nicefrac{1}{2} členů
            celostátního výboru.

        \item Celostátní výbor:
            \begin{enumerate}
            \item Schvaluje rozpočet a roční uzávěrku spolku.

            \item Ukládá úkoly předsednictvu spolku.

            \item Rozhoduje ve věcech, které nejsou svěřeny
                do působnosti jiných orgánů spolku.

            \item Schvaluje založení a zrušení krajské organizace.

            \item Na návrh předsednictva spolku určuje člena
                předsednictva spolku jako statutárního zástupce 
                s~oprávněním jednat jménem spolku samostatně a
                s~oprávněním k~dispozici s~účtem spolku
                u~peněžního ústavu.

            \item Jestliže se v průběhu funkčního období uvolní
                funkce 1. místopředsedy nebo místopředsedy spolku,
                volí a odvolává ze svého středu nástupce do konce
                funkčního období.

            \end{enumerate}

        \end{enumerate}

    \item Předsednictvo spolku:
        \begin{enumerate}
        \item Je výkonným orgánem spolku.

        \item Počet členů předsednictva spolku ustanovuje celostátní sjezd.

        \item Uvolní-li se funkce předsedy spolku nebo předseda
            spolku nemůže vykonávat svoji funkcí, je funkcí pověřen
            1.~místopředseda spolku.

        \item Předseda spolku, 1.~místopředseda a celostátním
            výborem určený člen předsednictva spolku jsou
            statutárními zástupci spolku, jsou oprávněni jednat
            jménem spolku samostatně a jsou oprávněni k~dispozici
            s~účtem spolku u~peněžního ústavu.

        \item Předsednictvo spolku je svoláváno předsedou spolku, nebo
            \nicefrac{1}{3} členů předsednictva spolku.

        \item Podmínkou usnášeníschopnosti předsednictva spolku je
            nadpoloviční většina hlasů všech členů předsednictva spolku.

        \item Předsednictvo spolku schvaluje přijetí členů spolku.

        \end{enumerate}

    \item Kontrolní komise:
        \begin{enumerate}
        \item Kontrolní komise se skládá nejméně ze tří členů.

        \item Podmínkou usnášeníschopnosti kontrolní komise je nadpoloviční většina
            hlasů všech členů kontrolní komise.

        \item Kontrolní komise rozhoduje o podané stížnosti nejpozději do
            30 dnů od data podání.

        \item Kontrolní komise je oprávněna kontrolovat veškerou činnost
            spolku, zejména hospodaření, dodržování stanov a naplňování cílů
            spolku. Jsou jí proto na požádání zpřístupněny veškeré dokumenty
            o~činnosti spolku.
            
        \item Člen kontrolní komise nesmí být členem jiného celostátního orgánu spolku.
        \end{enumerate}

    \item Krajská organizace a krajská konference:
        \begin{enumerate}
        \item Krajská organizace působí v rámci samosprávných krajů ČR.

        \item Minimální počet členů krajské organizace je 5.

        \item O založení krajské organizace rozhoduje celostátní výbor
            na návrh nejméně 5 členů spolku, kteří hodlají vyvíjet činnost
            v~krajské organizaci.

        \item Nejvyšším orgánem krajské organizace je krajská konference a
            schází se nejméně jednou za rok.

        \item Krajskou konferenci svolává krajský výbor, nebo
            \nicefrac{1}{3}~členů krajské organizace svým podpisem, a to
            nejméně 20 dnů před termínem konání.

        \item Krajská konference volí předsedu, jednoho nebo více
            místopředsedů a případné další členy krajského výboru.
            Volba do tohoto orgánu je tajná.

        \item Krajská konference volí delegáty celostátního sjezdu. Nezvolení
            kandidáti se automaticky stávají náhradníky v~pořadí dle výsledku
            volby.

        \item Podmínkou usnášeníschopnosti krajské konference je přítomnost
            nejméně \nicefrac{1}{3} členů krajské organizace.
        \end{enumerate}

    \item Krajský výbor:
        \begin{enumerate}
        \item Krajský výbor se skládá nejméně z~3 členů.

        \item Krajský výbor se schází nejméně jednou za 3 měsíce.

        \item Krajský výbor řídí a koordinuje činnost krajské organizace.

        \item Krajský výbor podává předsednictvu spolku stanovisko k~přijetí
            nových členů z~příslušného kraje.

        \item Krajský výbor svolává krajskou konferenci dle ustanovení
            Čl.~4.~odst.~7.~písm.~e).

        \item Krajský výbor schvaluje založení a zrušení
            okresní a místní organizace.

        \item Krajský výbor dohlíží na činnost okresních a místních organizací v kraji.

        \end{enumerate}

    \item Okresní organizace a okresní konference:
        \begin{enumerate}
        \item Okresní organizace je dobrovolným sdružením příslušných
            členů krajské organizace v~rámci okresu daného kraje. Minimální počet členů okresní organizace
            je 5.

        \item O založení okresní organizace rozhoduje krajský výbor na
            návrh nejméně 5~členů spolku, kteří hodlají vyvíjet činnost
            v~okresní organizaci.

        \item Nejvyšším orgánem okresní organizace je okresní konference a
            schází se nejméně jednou za rok.

        \item Okresní konferenci svolává okresní výbor,
            nebo \nicefrac{1}{3}~členů okresní organizace svým podpisem, a to
            nejméně 20 dnů před termínem konání.

        \item Okresní konference volí předsedu, jednoho nebo více místopředsedů a případné další členy
            okresního výboru.
            Volba do tohoto orgánu je tajná.

        \item Podmínkou usnášeníschopnosti okresní konference
            je přítomnost nejméně \nicefrac{1}{3} členů okresní organizace.
        \end{enumerate}

    \item Okresní výbor:
        \begin{enumerate}
        \item Okresní výbor se skládá nejméně ze 3 členů.

        \item Okresní výbor se schází nejméně jednou za 3 měsíce.

        \item Okresní výbor řídí a koordinuje činnost okresní organizace.

        \item Okresní výbor schvaluje žádost o přijetí člena krajské organizace do okresní organizace.

        \item Okresní výbor svolává okresní konferenci dle ustanovení Čl.~4.~odst.~9.~písm.~d).

        \item Předseda okresního výboru se účastní krajského výboru s hlasem poradním.
        \end{enumerate}

    \item Místní organizace a výroční členská schůze:
        \begin{enumerate}
        \item Místní organizace je dobrovolným sdružením místně příslušných
            členů krajské organizace. Minimálním počet členů místní organizace
            je 3.

        \item O založení místní organizace rozhoduje krajský výbor na
            návrh nejméně 3~členů spolku, kteří hodlají vyvíjet činnost
            v~místní organizaci.

        \item Nejvyšším orgánem místní organizace je výroční členská schůze a
            schází se nejméně jednou za rok.

        \item Výroční členskou schůzi svolává předsednictvo místní organizace,
            nebo \nicefrac{1}{3}~členů místní organizace svým podpisem, a to
            nejméně 20 dnů před termínem konání.

        \item Výroční členská schůze volí předsedu a místopředsedy místní
            organizace.
            Volba do tohoto orgánu je tajná.

        \item Podmínkou usnášeníschopnosti výroční členské schůze
            je přítomnost nejméně \nicefrac{1}{3} členů místní organizace.
        \end{enumerate}


    \item Předsednictvo místní organizace:
        \begin{enumerate}
        \item Předsednictvo místní organizace se schází nejméně jednou za
            3~měsíce.

        \item Předsednictvo místní organizace řídí a koordinuje její činnost.

        \item Předsednictvo místní organizace schvaluje žádost o~přijetí
            člena krajské organizace do místní organizace.

        \item Předsednictvo místní organizace svolává výroční členskou
            schůzi dle ustanovení Čl.~4.~odst.~11.~písm.~d).

        \item Předseda místní organizace se účastní krajského a okresního výboru
            s~hlasem poradním.
        \end{enumerate}
    \end{enumerate}



\section{Zásady hospodaření}
    \begin{enumerate}
    \item Spolek může nabývat majetek, užívat jej a disponovat s~ním.

    \item Příjmy spolku tvoří zejména
        \begin{enumerate}
        \item dary právnických a fyzických osob, dotace a granty,
        \item příjmy plynoucí z~hospodaření s~majetkem spolku,
        \item členské příspěvky.
        \end{enumerate}

    \item Za hospodaření spolku odpovídá předsednictvo spolku, které
        rovněž disponuje finančními prostředky spolku a je oprávněno vymáhat
        plnění závazků třetích osob vůči spolku všemi právními prostředky.
        Za kontrolu hospodaření s~majetkem odpovídá předseda kontrolní komise.

    \item Statutární zástupci mohou samostatně rozhodovat o~finančních
        záležitostech do výše 50.000,-~Kč (padesát tisíc korun českých).

    \item Spolek může vynakládat své prostředky zejména na své vybavení,
        vzdělávání členů a pořádání akcí v~souladu s~cíli činnosti spolku.
    \end{enumerate}



\section{Zánik spolku, majetkové vypořádání}
    \begin{enumerate}
    \item Rozhodne-li celostátní sjezd o~zániku spolku nebo sloučení s~jiným
        spolkem v souladu s ustanovením Čl.~4.~odst.~3.~písm.~f), anebo
        nastane-li zánik v~důsledku pravomocného rozhodnutí státního orgánu
        o~rozpuštění spolku, předsednictvo spolku ustanoví likvidační
        komisi. Likvidační komise je nejméně tříčlenná a ze svého středu si 
        volí likvidátora. Ten postupuje v~souladu s~platným právem obdobně
        podle ustanovení o~likvidaci právnických osob, přičemž činnost
        likvidátora směřuje k~provedení všech úkonů nezbytných k~likvidaci a
        majetkovému vypořádání.
    \end{enumerate}



\section{Závěrečná a přechodná ustanovení}
    \begin{enumerate}
    \item Tyto stanovy nabývají účinnosti dnem jejich schválením celostátním sjezdem.

    \item Schválením těchto stanov se ruší stanovy spolku schválené na
        Valné hromadě konané dne 9.~3.~2013 v~Praze.

    \end{enumerate}

\end{document}
