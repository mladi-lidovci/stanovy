\documentclass[a4paper]{article}

\usepackage[english,czech]{babel}
\usepackage[T1]{fontenc}
\usepackage[utf8]{inputenc}
\usepackage{cmap}
\usepackage{lmodern}
\usepackage[final]{pdfpages}
\usepackage{idxlayout}

\ifpdf
    \usepackage[protrusion,expansion,step=1]{microtype}
\else
    \usepackage[protrusion]{microtype}
\fi

\usepackage{makeidx}
\usepackage{units}

\makeindex

\usepackage[
bookmarks=true,
colorlinks=true,
breaklinks=true,
urlcolor=black,
citecolor=black,
linkcolor=black,
unicode=true,
]
{hyperref}

\usepackage[square, numbers]{natbib}
\usepackage{url}


\RequirePackage[left=1in,right=1in,top=1in,bottom=1in]{geometry}
\usepackage{enumerate}


\newcommand{\halfspace}{\hskip 0.4ex}
\newcommand{\mlname}{Mladí lidovci, o.\halfspace{}s.}


\usepackage[explicit]{titlesec}
\titleformat{\section}
  {\normalfont\Large\filcenter}{}{0em}{\S\halfspace\arabic{section}\halfspace\hspace*{ 0.3em}#1}
\titleformat{name=\section,numberless}
  {\normalfont\Large\filcenter}{}{0em}{\MakeUppercase{#1}}

\renewcommand*{\thesection}{\Roman{section}}



\title{\normalfont\Huge{Stanovy \mlname}}
\date{}
\author{}

\begin{document}

\maketitle

\section{Základní ustanovení}
    \begin{enumerate}
    \item \mlname{} jsou občanským sdružením, řádně registrovaným
        u~Ministerstva vnitra České republiky a vyvíjejícím činnost v~souladu
        se zákonem č. 83/1990 Sb., o~sdružování občanů, ve znění pozdějších
        předpisů.

    \item Úplný název občanského sdružení je \uv{\mlname} (dále také
        sdružení). V~případech, kdy je možné tak učinit, sdružení může
        vystupovat pod názvem \uv{Mladí lidovci} nebo zkratkou \uv{ML}.
        Anglický překlad názvu sdružení je \uv{Youth people's party}, zkratkou
        \uv{YPP}, německý překlad je \uv{Junge Volkspartei}, zkratkou \uv{JV}.

    \item Existence \mlname{} je založena na přijetí odpovědnosti za
        konstruktivní utváření budoucnosti Evropy. Činí tak s~vědomím
        kulturního a sociálního bohatství vytvořeného během staletí trvajících
        křesťanských dějin Evropy, které nikde jinde ani později již nenašlo
        obdoby. Proto se svojí prací zasazuje o~uchování společného duchovního
        dědictví a křesťanské zásady si bere za základ své činnosti. \mlname{}
        úzce spolupracující s~partnerskými občanskými sdruženími v~České
        republice i zahraničními organizacemi, nabízí tak solidní podporu
        nejen křesťanům, ale všem mladým lidem, zejména politicky angažovaným,
        kteří se chtějí aktivně zasazovat o~lepší budoucnost České republiky
        a Evropy.

    \item Adresa sídla sdružení: Palác Charitas, Karlovo náměstí 317/5,
        128~01~Praha~2.
    \end{enumerate}



\section{Poslání, cíle a činnost}
    \begin{enumerate}
    \item Posláním sdružení je vytváření prostředí stimulujícího osobnostní,
        profesní, sociální, politický a duchovní růst mladého člověka.

    \item Zvláštní zřetel je kladen na růst společensko-politický.

    \item Cíle sdružení, orientované na mladé lidi, studenty a absolventy
        vysokých škol, jsou zejména následující:
        \begin{enumerate}
            \item podporovat vzdělání v~otázkách křesťansko-konzervativní
                politiky, s~důrazem ukotvení hodnot založených na evropské
                křesťanské morálce,

            \item naplňovat společenské potřeby cílové skupiny,

            \item podporovat vzájemnou úctu, ohleduplnost a nezištnou službu
                v~rámci aktivního působení ve veřejném životě,

            \item iniciovat rozvoj schopností a dovedností umožňujících hlubší
                pochopení politické morální odpovědnosti,

            \item podporovat zvýšení informovanosti o~politickém a kulturním
                dění ve světě, obzvláště v~České republice a v~Evropské unii,

            \item pomáhat rozvíjet schopnost zaujímat a vyjadřovat stanoviska
                ke společenským otázkám.
        \end{enumerate}

    \item Činnost sdružení spočívá zejména v~následujícím:
        \begin{enumerate}
        \item vytváření společenství mladých lidí se zájmem o~politické dění,

        \item organizování a zprostředkovávání vzdělávacích akcí, přednášek,
            seminářů, stáží a jiných aktivit potřebných k formaci osobnosti,

        \item pořádání kulturních a společenských akcí.
        \end{enumerate}
    \end{enumerate}



\section{Členství}
    \begin{enumerate}
    \item Členem sdružení může být osoba starší 15 let a mladší 35 let věku,
        pokud není členem žádné politické strany s~výjimkou politické strany
        KDU-ČSL nebo zahraniční politické strany založené na
        křesťansko-demokratických principech.

    \item K~přijetí za člena sdružení dochází na základě podání přihlášky a
        jejího schválení předsednictvem sdružení.

    \item Každý člen sdružení je zařazen do jedné krajské organizace, toto
        zařazení schvaluje předsednictvo sdružení.

    \item Pokud je přihláška o~členství předsednictvem sdružení zamítnuta
        musí být uchazeč písemně informován nejpozději do 30 pracovních dnů.
        Uchazeč o~členství, jehož přihláška byla zamítnuta, může podat
        přihlášku znovu nejdříve po uplynutí 6 měsíců ode dne zamítnutí
        přihlášky.

    \item Členství ve sdružení zaniká:
        \begin{enumerate}
        \item úmrtím,
        \item dovršením věku 35 let,
        \item písemným prohlášením o vystoupení doručeným předsednictvu
            sdružení,
        \item poruším podmínky členství stanov sdružení (ustanovení \S~3.~1.)
        \item nezaplacením členského příspěvku,
        \item vyloučením.
        \end{enumerate}

    \item O~vyloučení člena rozhoduje předsednictvo sdružení. Vyloučit člena
        sdružení lze pouze pro:
        \begin{enumerate}
        \item hrubé porušení stanov,
        \item dlouhodobé zanedbávání členských povinností,
        \item hrubé a zaviněné porušení povinností při hospodaření s~majetkem
            sdružení,
        \item spáchání úmyslného trestného činu, za který byl odsouzen,
        \item opakované jednání, které poškozuje sdružení.
        \end{enumerate}

    \item Proti rozhodnutí o~vyloučení se může člen sdružení odvolat ke
        kontrolní komisi do 15 dnů ode dne doručení písemného vyhotovení
        rozhodnutí. V~případě podání odvolání vyloučeným členem se účinnost
        rozhodnutí o~vyloučení pozastavuje až do rozhodnutí kontrolní komise.

    \item Člen je povinen platit členské příspěvky ve výši stanovené
        předsednictvem sdružení. O~výši členských příspěvků může rozhodnout
        i~valná hromada.
    \end{enumerate}



\section{Organizační struktura}
    \begin{enumerate}
    \item Organizační struktura sdružení se skládá z~valné hromady,
        předsednictva sdružení, krajských organizací a kontrolní komise.

    \item Valná hromada:
        \begin{enumerate}
        \item Valná hromada je nejvyšší orgán sdružení a schází se nejméně
            jednou za rok.

        \item Valnou hromadu svolává předsednictvo sdružení, nebo 50 členů
            sdružení svým podpisem, a to nejméně 20 dnů před termínem konání
            valné hromady.

        \item Členy valné hromady jsou všichni platní členové sdružení v~den
            konání valné hromady.

        \item Výlučnou pravomocí valné hromady je volba a odvolání
            předsednictva sdružení, změna stanov a volba členů a předsedy
            kontrolní komise. Volba do těchto orgánů sdružení je tajná.

        \item Podmínkou platnosti voleb nebo hlasování valné hromady je
            přítomnost nejméně \nicefrac{1}{3} členů sdružení a prostá většina
            hlasů, není-li stanovami stanoveno jinak.

        \item Valná hromada může rozhodnout o~zrušení sdružení nebo jeho
            sloučení s~jiným sdružením, a to dvoutřetinovou většinou hlasů
            všech svých členů.
        \end{enumerate}

    \item Předsednictvo sdružení:
        \begin{enumerate}
        \item Je výkonným orgánem sdružení.

        \item Počet členů předsednictva sdružení ustanovuje valná hromada a
            následně volí předsedu sdružení, 1.~místopředsedu a ostatní členy
            předsednictva sdružení.

        \item Funkční období předsednictva sdružení je jednoleté, může skončit
            i~dříve zvolením nového předsednictva sdružení, pokud byla svolána
            valná hromada na termín před uplynutím funkčního období.

        \item Nemá-li sdružení předsedu sdružení nebo předseda sdružení nemůže
            vykonávat svoji funkci, předsednictvo sdružení zvolí ze svého
            středu člena, kterého pověří vykonáváním funkce předsedy sdružení.

        \item Předseda sdružení a jeho 1.~místopředseda jsou statutárními
            zástupci sdružení a jsou oprávnění jednat jménem sdružení
            samostatně. K~dispozici s~účtem sdružení u~peněžního ústavu jsou
            oprávněni jednat předseda sdružení, 1. místopředseda sdružení a
            předsedou pověřený člen předsednictva sdružení.

        \item Podmínkou platnosti hlasování předsednictva sdružení je
            nadpoloviční většina hlasů všech členů předsednictva sdružení.

        \item Předsednictvo sdružení schvaluje přijetí členů sdružení.

        \item Předsednictvo sdružení schvaluje založení a zrušení krajské,
            nebo místní organizace.
        \end{enumerate}

    \item Krajská organizace a krajská konference:
        \begin{enumerate}
        \item Krajská organizace působní v rámci samosprávných krajů ČR.

        \item Minimální počet členů krajské organizace je 5.

        \item O založení krajské organizace rozhoduje předsednictvo sdružení na
            návrh nejméně 5 členů sdružení, kteří hodlají vyvíjet činnost
            v~krajské organizaci.

        \item Nejvyšším orgánem krajské organizace je krajská konference a schází
            se nejméně jednou za rok.

        \item Krajskou konferenci svolává krajský výbor, nebo
            \nicefrac{1}{3}~členů krajské organizace svým podpisem, a to
            nejméně 20 dnů před termínem konání.

        \item Krajská konference volí předsedu, místopředsedy a případné další členy krajského výboru.
            Funkce předsedy krajské organizace je totožná s~funkcí předsedy
            krajského výboru.

        \item Krajská konference volí delegáty celostátního sjezdu. Nezvolení
            kandidáti se automaticky stávají náhradníky v pořadí dle výsledku
            volby.

        \item Podmínkou platnosti voleb krajské konference je přítomnost
            nejméně \nicefrac{1}{3} členů krajské organizace.
        \end{enumerate}

    \item Krajský výbor:
        \begin{enumerate}
        \item Krajský výbor se skládá nejméně z 3 členů.

        \item Krajský výbor se schází nejméně jednou za 3 měsíce.

        \item Krajský výbor řídí a koordinuje činnost krajské organizace.

        \item Krajský výbor podává předsednictvu sdružení stanovisko k~přijetí
            nových členů z~příslušného kraje.

        \item Krajský výbor svolává krajskou konferenci dle ustanovení
            \S~4.~odst.~6.~písm.~e).

        \item Krajský výbor dohlíží na činnost místních organizací v kraji.
        \end{enumerate}

    \item Místní organizace a výroční členská schůze:
        \begin{enumerate}
        \item Místní organizace je dobrovolným sdružením místně příslušných
            členů krajské organizace. Minimálním počet členů místní organizace
            je 3.

        \item O založení místní organizace rozhoduje předsednictvo sdružení na
            návrh nejméně 3 členů sdružení, kteří hodlají vyvíjet činnost
            v~místní organizaci.

        \item Nejvyšším orgánem místní organizace je výroční členská schůze a
            schází se nejméně jednou za rok.

        \item Výroční členskou schůzi svolává předsednictvo místní organizace,
            nebo \nicefrac{1}{3}~členů místní organizace svým podpisem, a to
            nejméně 20 dnů před termínem konání.

        \item Výroční členská schůze volí předsedu a místopředsedy místní
            organizace.

        \item Podmínkou platnosti voleb výroční členské schůze
            je přítomnost nejméně \nicefrac{1}{3} členů místní organizace.
        \end{enumerate}

    \item Předsednictvo místní organizace:
        \begin{enumerate}
        \item Předsednictvo místní organizace se schází nejméně jednou za
            3 měsíce.

        \item Předsednictvo místní organizace řídí a koordinuje její činnost.

        \item Předsednictvo místní organizace schvaluje žádost o~přijetí
            člena krajské organizace do místní organizace.

        \item Předsednictvo místní organizace svolává výroční členskou
            schůzi dle ustanovení \S~4.~odst.~8.~písm.~d).

        \item Předseda místní organizace se účastní krajského výboru
            s~hlasem poradním.
        \end{enumerate}

    \item Kontrolní komise je oprávněna kontrolovat veškerou činnost sdružení,
        zejména hospodaření, dodržování stanov a naplňování cílů sdružení.
        Jsou jí proto na požádání zpřístupněny veškeré dokumenty o~činnosti
        sdružení. Kontrolní komise kontroluje platnost voleb předsednictva
        sdružení. Skládá se nejméně ze tří členů, jejichž mandát je jednoletý
        a zaniká zvolením nových členů kontrolní komise. Podmínkou platnosti
        hlasování kontrolní komise je nadpoloviční většina hlasů všech členů
        kontrolní komise.
    \end{enumerate}



\section{Zásady hospodaření}
    \begin{enumerate}
    \item Sdružení může nabývat majetek, užívat jej a disponovat s ním.

    \item Příjmy sdružení tvoří zejména
        \begin{enumerate}
        \item dary právnických a fyzických osob, dotace a granty,
        \item příjmy plynoucí z~hospodaření s~majetkem sdružení,
        \item členské příspěvky.
        \end{enumerate}

    \item Za hospodaření sdružení odpovídá předsednictvo sdružení, které
        rovněž disponuje finančními prostředky sdružení a je oprávněna vymáhat
        plnění závazků třetích osob vůči sdružení všemi právními prostředky.
        Za kontrolu hospodaření s~majetkem odpovídá předseda kontrolní komise.

    \item Statutární zástupci mohou samostatně rozhodovat o~finančních
        záležitostech do výše 50.000,- Kč (padesát tisíc korun českých).

    \item Sdružení může vynakládat své prostředky zejména na své vybavení,
        vzdělávání členů a pořádání akcí v~souladu s~cíli činnosti sdružení.
    \end{enumerate}



\section{Zánik sdružení, majetkové vypořádání}
    \begin{enumerate}
    \item Rozhodne-li valná hromada o~zániku sdružení nebo sloučení s~jiným
        sdružením v souladu s ustanovením \S~4.~2.~f.), anebo nastane-li zánik
        v~důsledku pravomocného rozhodnutí státního orgánu o~rozpuštění
        sdružení, předsednictvo sdružení ustanoví likvidační komisi.
        Likvidační komise je nejméně tříčlenná a ze svého středu si zvolí
        likvidátora. Ten postupuje v~souladu s~platným právem obdobně podle
        ustanovení o~likvidaci právnických osob, přičemž činnost likvidátora
        směřuje k~provedení všech úkonů nezbytných k~likvidaci a majetkovému
        vypořádání.
    \end{enumerate}



\end{document}
